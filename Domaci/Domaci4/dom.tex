\documentclass[12pt]{article}
\usepackage{geometry}
    \geometry{
        a4paper,
        lmargin=2cm,
        rmargin=2cm,
        tmargin=2cm,
        bmargin=2cm
    }
\usepackage{float}
\usepackage{indentfirst}
\usepackage{caption}
\usepackage{graphicx}
\usepackage{fontspec}
\usepackage{listings}
\usepackage{color}
\usepackage[T1]{fontenc}
\lstloadlanguages{C,C++,csh,Java}

\setmainfont[Mapping=tex-text]{Times New Roman}
\definecolor{red}{rgb}{0.6,0,0}
\definecolor{blue}{rgb}{0,0,0.6}
\definecolor{green}{rgb}{0,0.8,0}
\definecolor{cyan}{rgb}{0.0,0.6,0.6}
\renewcommand{\contentsname}{Sadržaj}

\lstset{
language=csh,
basicstyle=\footnotesize\ttfamily,
numbers=left,
numberstyle=\tiny,
numbersep=5pt,
frame=box,
tabsize=2,
extendedchars=true,
breaklines=true,
stringstyle=\color{blue}\ttfamily,
showspaces=false,
showtabs=false,
xleftmargin=17pt,
framexleftmargin=17pt,
framexrightmargin=5pt,
framexbottommargin=4pt,
commentstyle=\color{green},
morecomment=[l]{//}, %use comment-line-style!
morecomment=[s]{/*}{*/}, %for multiline comments
showstringspaces=false,
morekeywords={ abstract, event, new, struct,
as, explicit, null, switch,
base, extern, object, this,
bool, false, operator, throw,
break, finally, out, true,
byte, fixed, override, try,
case, float, params, typeof,
catch, for, private, uint,
char, foreach, protected, ulong,
checked, goto, public, unchecked,
class, if, readonly, unsafe,
const, implicit, ref, ushort,
continue, in, return, using,
decimal, int, sbyte, virtual,
default, interface, sealed, volatile,
delegate, internal, short, void,
do, is, sizeof, while,
double, lock, stackalloc,
else, long, static,
enum, namespace, string},
keywordstyle=\color{cyan},
identifierstyle=\color{black},
backgroundcolor=\color{white},
}

\title{Domaci 3 - kriptografija}
\author{Ognjen Čavić E2 161/2024}
\date{Novembar 2024}
\begin{document}
\maketitle
\section{Prvi zadatak}
\par Prvi zadatak je šifrovati frazu \textbf{VENI VIDI VICI} pomoću
\textit{Playfair} algoritma.
Ponuđeni ključevi, tj reči kojima se matrica (ili tabela) slova može pomešati,
su \textbf{IRITIRATI}, \textbf{ANANAS} i \textbf{SUNCE}.
U opštem slučaju, cilj je da se za ključ odabere reč sa što više različitih slova
jer to rezultuje tabelom koja je više izmešana.
Stoga je najbolja reč od ponuđenih \textbf{SUNCE} jer ima najviše različitih
slova čak iako je najkraća i time se dobija najmanje predvidiva tabela.
\begin{center}
\begin{table}[h]
	\centering
    \begin{tabular}{|c|c|c|c|c|}
        \hline
		S & U & N & C & E \\
		\hline
		A & B & D & F & G \\
		\hline
		H & I & K & L & M \\
		\hline
		O & P & Q & R & T \\
		\hline
		V & W & X & Y & Z \\
		\hline
    \end{tabular}
	\caption*{Tabela 1: Matrica slova \textit{Playfair} algoritma kada se
	koristi reč \textbf{SUNCE}}
\end{table}
\end{center}
Nakon što se fraza podeli u segmente od po dva slova koji se šifruju na osnovu
matrice, dobija se konačna poruka:
\begin{center}
\begin{table}[ht]
    \centering
    \begin{tabular}{|c|c|c|c|c|c|c|}
		\hline
		Segmenti originalne poruke & VE & NI & VI & DI & VI & CI \\
		\hline
		Segmenti šifrovane poruke & ZS & UK & WH & BK & WH & UL \\
		\hline
    \end{tabular}
	\caption*{Tabela 2: Odgovarajući segmenti šifrovane i originalne poruke}
\end{table}
\end{center}
koja glasi \textbf{ZSUKWHBKWHUL}.
\newpage
\section{Drugi zadatak}
\par Dok se prvi zadatak bavio šifrovanjem poruke, cilj drugog jeste da
dešifruje primljenu poruku ukoliko je ključ poznat.
Dodatna komplikacija je to što koristi ćirilično pismo, koja sadrži 30 slova i
zahteva produženje tabele.
Fraza koju treba dešifrovati je \textbf{ОЗМЕЖАУЛМОЛИЛПНХОЏ} dok ključ
koji se koristi je \textbf{КОЛОКВИЈУМ} i nakon uklanjanja svih slova koja
se ponavljaju dobija se \textbf{КОЛВИЈУМ}.
Na osnovu ključa se formira matrica slova.
\begin{center}
\begin{table}[h]
	\centering
    \begin{tabular}{|c|c|c|c|c|c|}
        \hline
		К & О & Л & В & И & Ј \\
		\hline
		У & М & А & Б & Г & Д \\
		\hline
		Ђ & Е & Ж & З & Љ & Н \\
		\hline
		Њ & П & Р & С & Т & Ћ \\
		\hline
		Ф & Х & Ц & Ч & Џ & Ш \\
		\hline
    \end{tabular}
	\caption*{Tabela 3: Tabela \textit{Playfair} algortima za ćirilična slova}
\end{table}
\end{center}
Postupak dešifrovanja je isti kao i postupak šifrovanja sa tim da za slova
u istom redu ili koloni pomera se ulevo umesto udesno i na gore umesto na dole,
respektivno.
Nakon što je sve potrebno za dešifrovanje poznato, preostaje samo to i odraditi:
\begin{center}
\begin{table}[ht]
	\centering
	\begin{tabular}{|c|c|c|c|c|c|c|c|c|c|}
		\hline
		Segmenti šifrovane poruke & ОЗ & МЕ & ЖА & УЛ & МО & ЛИ & ЛП & НХ & ОЏ \\
		\hline
		Segmenti dešifrovane poruke & ВЕ & ОМ & АЛ & АК & О\textcolor{red}{Х}
									& ОВ & ОР & ЕШ & ИХ \\
		\hline
	\end{tabular}
	\caption*{Tabela 4: Odgovarajući segmenti šifrovane i originalne poruke}
\end{table}
\end{center}
nakon odbacivanja obojenog \textcolor{red}{\textbf{Х}} koje je umetnuto kako bi
se izbeglo da se dva ista slova nalaze u segmentu, dobija se konačna poruka
\textbf{ВЕОМА ЛАКО ОВО РЕШИХ}.
\end{document}

