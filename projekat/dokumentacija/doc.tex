\documentclass[12pt]{article}
\usepackage{geometry}
    \geometry{
        a4paper,
        lmargin=2cm,
        rmargin=2cm,
        tmargin=2cm,
        bmargin=2cm
    }
\usepackage{float}
\usepackage{indentfirst}
\usepackage{caption}
\usepackage{graphicx}
\usepackage{fontspec}
\usepackage{listings}
\usepackage{color}
\usepackage[T1]{fontenc}
\lstloadlanguages{C,C++,csh,Java}

\setmainfont[Mapping=tex-text]{Times New Roman}
\definecolor{red}{rgb}{0.6,0,0}
\definecolor{blue}{rgb}{0,0,0.6}
\definecolor{green}{rgb}{0,0.8,0}
\definecolor{cyan}{rgb}{0.0,0.6,0.6}
\renewcommand{\contentsname}{Sadržaj}

\lstset{
language=csh,
basicstyle=\footnotesize\ttfamily,
numbers=left,
numberstyle=\tiny,
numbersep=5pt,
frame=box,
tabsize=2,
extendedchars=true,
breaklines=true,
stringstyle=\color{blue}\ttfamily,
showspaces=false,
showtabs=false,
xleftmargin=17pt,
framexleftmargin=17pt,
framexrightmargin=5pt,
framexbottommargin=4pt,
commentstyle=\color{green},
morecomment=[l]{//}, %use comment-line-style!
morecomment=[s]{/*}{*/}, %for multiline comments
showstringspaces=false,
morekeywords={ abstract, event, new, struct,
as, explicit, null, switch,
base, extern, object, this,
bool, false, operator, throw,
break, finally, out, true,
byte, fixed, override, try,
case, float, params, typeof,
catch, for, private, uint,
char, foreach, protected, ulong,
checked, goto, public, unchecked,
class, if, readonly, unsafe,
const, implicit, ref, ushort,
continue, in, return, using,
decimal, int, sbyte, virtual,
default, interface, sealed, volatile,
delegate, internal, short, void,
do, is, sizeof, while,
double, lock, stackalloc,
else, long, static,
enum, namespace, string},
keywordstyle=\color{cyan},
identifierstyle=\color{black},
backgroundcolor=\color{white},
}

\title{Projektni zadatak - Konzistencija}
\author{Ognjen Čavić E2 161/2024}
\date{\today}
\begin{document}
\maketitle
\tableofcontents
\break
\section{Uvod}
Potrebno je napraviti sistem gde postoji 10 senzora koji na nasumičnom itervalu
generišu nasumične vrednosti i "menadžera" koji poravnava te vrednosti tj. bira
jednu od tih vrednosti koja će biti proglašena za glavnu. Ovaj sistem se može
opisati kao skup od 3 glavne komponente:
\begin{itemize}
	\item \textbf{Simulator} - Pokreće po jednu nit za svaki od senzora, koji generiše
		novu vrednost na vremenskom intervalu između 1 i 10 sekundi i upisuje je
		u sopstvenu tabelu u bazi podataka pomoću \textit{Entity Framework}-a.
	\item \textbf{Menadžer}- Na svakih 60 sekundi vrši poravnanje, tj. bira 
		čitanje senzora koje je najbolje i koje će biti upisano u tabele svih 
		senzora.
		Kriterijum najbolje vrednosti je ono koje je najskorije u prihvatljivom
		opsegu $\pm5^oC$. Ukoliko ono ne postoji samo uzeti najskorije.
	\item \textbf{WCF servis} - Definiše način komunikacije između menadžera i
		senzora.
\end{itemize}
\section{Organizacija rešenja}
Rešenje se sastoji od 4 manja projekta, gde svaki predstavlja jednu od celina
koja je prethodno navedena, sa tim da dodati četvrti projekat opisuje način
struktuiranja podataka za upis u bazu:
\begin{itemize}
	\item ConsistencyManager - menadžer koji vrši poravnanje
	\item ConsistencyService - WCF servis
	\item Sensors - pokretanje niti koje simuliraju senzore
	\item SensorsDatabaseContext - kontekst baze podataka
\end{itemize}
\subsection{ConsistencyManager}
\subsubsection{ConsistencyManager/Program.cs}
Definiše dve funkcije, \textbf{Main} i \textbf{findBest}.
\textbf{Main} funkcija na svakih 60 sekundi šalje upit bazi podataka za 
poslednji podatak iz svake tabele, pomoću funkcije \textbf{findBest} određuje
koji od senzora je dao najbolju vrednost i upisuje je u svaku od tabela.
Potrebno je napomenuti da menadžer instancira \textit{subscribera} i 
\textit{publishera}, koje koristi za upite i upise u bazu, respektivno.
\subsection{ConsistencyService}
\subsubsection{ConsistencyService/ConsistencyService.cs}

\end{document}
