\documentclass[12pt]{article}
\usepackage{geometry}
    \geometry{
        a4paper,
        lmargin=2cm,
        rmargin=2cm,
        tmargin=2cm,
        bmargin=2cm
    }
\usepackage{float}
\usepackage{indentfirst}
\usepackage{caption}
\usepackage{graphicx}
\usepackage{fontspec}
\usepackage{hyperref}
\usepackage{listings}
\usepackage{color}
\usepackage[T1]{fontenc}
\lstloadlanguages{C,C++,csh,Java}

\setmainfont[Mapping=tex-text]{Times New Roman}
\definecolor{red}{rgb}{0.6,0,0}
\definecolor{blue}{rgb}{0,0,0.6}
\definecolor{green}{rgb}{0,0.8,0}
\definecolor{cyan}{rgb}{0.0,0.6,0.6}
\renewcommand{\contentsname}{Sadržaj}

\lstset{
language=csh,
basicstyle=\footnotesize\ttfamily,
numbers=left,
numberstyle=\tiny,
numbersep=5pt,
frame=box,
tabsize=2,
extendedchars=true,
breaklines=true,
stringstyle=\color{blue}\ttfamily,
showspaces=false,
showtabs=false,
xleftmargin=17pt,
framexleftmargin=17pt,
framexrightmargin=5pt,
framexbottommargin=4pt,
commentstyle=\color{green},
morecomment=[l]{//}, %use comment-line-style!
morecomment=[s]{/*}{*/}, %for multiline comments
showstringspaces=false,
morekeywords={ abstract, event, new, struct,
as, explicit, null, switch,
base, extern, object, this,
bool, false, operator, throw,
break, finally, out, true,
byte, fixed, override, try,
case, float, params, typeof,
catch, for, private, uint,
char, foreach, protected, ulong,
checked, goto, public, unchecked,
class, if, readonly, unsafe,
const, implicit, ref, ushort,
continue, in, return, using,
decimal, int, sbyte, virtual,
default, interface, sealed, volatile,
delegate, internal, short, void,
do, is, sizeof, while,
double, lock, stackalloc,
else, long, static,
enum, namespace, string},
keywordstyle=\color{cyan},
identifierstyle=\color{black},
backgroundcolor=\color{white},
}

\title{Projektni zadatak - Konzistencija}
\author{Ognjen Čavić E2 161/2024}
\date{\today}
\begin{document}
\maketitle
\tableofcontents
\break
\section{Uvod}
Potrebno je napraviti sistem u kojem 10 senzora na nasumičnim vremenskim itervalima
generišu nasumične vrednosti, dok "menadžer" vrši njihovo poravnjanje, tj. bira
jednu od tih vrednosti koja će biti proglašena glavnom.
Ovaj sistem se može opisati kao skup od 3 glavne komponente:
\begin{itemize}
	\item \textbf{Simulator} - Pokreće po jednu nit za svaki senzor.
		Svaka nit na vremenskom intervalu između 1 i 10 sekundi generiše novu
		vrednost i  upisuje je u sopstvenu tabelu u bazi podataka pomoću 
		\textit{Entity Framework}-a.
	\item \textbf{Menadžer}- Na svakih 60 sekundi vrši poravnanje, tako što bira 
		čitanje senzora koje je najpovoljnije i upisuje u tabele svih senzora.
		Kriterijum za izbor najbolje vrednosti je najskorije očitavanje u 
		prihvatljivom opsegu $\pm5^oC$.
		Ukoliko takvo ne postoji uzima se najskorije.
	\item \textbf{WCF servis} - Definiše način komunikacije između menadžera i
		senzora.
\end{itemize}
\section{Organizacija rešenja}
Rešenje se sastoji od 4 manja projekta, gde svaki predstavlja jednu od prethodno
navedenih celina, sa tim da dodati četvrti projekat opisuje način
struktuiranja podataka za upis u bazu:
\begin{itemize}
	\item ConsistencyService - WCF servis
	\item ConsistencyManager - menadžer koji vrši poravnanje
	\item Sensors - pokretanje niti koje simuliraju senzore
	\item SensorsDatabaseContext - kontekst baze podataka
\end{itemize}
\subsection{ConsistencyService}
\subsubsection{ConsistencyService/ConsistencyService.cs} \label{sec:servicecs}
Klasa \textbf{TemperatureInfo} opisuje oblik podataka koji se razmenjuju između
korisnika servisa i formiraju ga 4 polja:
\begin{enumerate}
	\item \textbf{sensor\_id} - Identifikacioni broj senzora, broj od 1 do 10 
	\item \textbf{temperature} - Vrednost temperature
	\item \textbf{timestamp} - vremenski podatak
	\item \textbf{source} - izvor podatka, tj. ko ga upisuje u bazu, senzor ili
		menadžer tokom poravnanja
\end{enumerate}
U ovom fajlu su samo još definisani interfejsi servisa. \textbf{IPublisher}
definiše metod za upis u bazu, dok interfejs \textbf{ISubscriber} definiše metod
za slanje upita.
\subsubsection{ConsistencyService/ConsistencyService.svc.cs}
Implementacija \textbf{IPublisher} i \textbf{ISubscriber} interfejsa unutar
\textbf{ConsistencyService} klase.
Bitno je primetiti da klasa nasleđuje oba interfejsa.
Metoda \textbf{publishTemp} upisuje zadati \textbf{TemperatureInfo}, što se
koristi da i menadžer i senzori upisuju u bazu, tako što \textbf{source} polje
postave na odgovarajuću vrednost.
Poslednje očitavanje senzora se dobija pozivom \textbf{querySensor} i
prosleđivanjem odgovarajućeg rednog broja.


\subsection{ConsistencyManager}
\subsubsection{ConsistencyManager/Program.cs}
Sadrži dve funkcije, \textbf{Main} i \textbf{findBest}.
\textbf{Main} funkcija na svakih 60 sekundi šalje upit bazi podataka za 
poslednji podatak iz svake tabele, potom pomoću funkcije \textbf{findBest}
određuje koje od čitanja senzora je bilo najbolje.
Nakon toga se izmenjuju polja \textbf{source} i \textbf{timestamp} tako da se
označi da je podatak upisan kao rezultat poravnanja i vreme kada je on upisan,
takav podatak se upisuje u svaku od tabela.
Potrebno je napomenuti da menadžer instancira \textit{subscribera} i 
\textit{publishera}, koje koristi za upite i upise u bazu, respektivno.
\end{document}
