\documentclass[12pt]{article}
\usepackage{geometry}
    \geometry{
        a4paper,
        lmargin=2cm,
        rmargin=2cm,
        tmargin=2cm,
        bmargin=2cm
    }
\usepackage{float}
\usepackage{indentfirst}
\usepackage{caption}
\usepackage{graphicx}
\usepackage{fontspec}
\usepackage{hyperref}
\usepackage{listings}
\usepackage{color}
\usepackage[T1]{fontenc}
\lstloadlanguages{C,C++,csh,Java}

\setmainfont[Mapping=tex-text]{Times New Roman}
\definecolor{red}{rgb}{0.6,0,0}
\definecolor{blue}{rgb}{0,0,0.6}
\definecolor{green}{rgb}{0,0.8,0}
\definecolor{cyan}{rgb}{0.0,0.6,0.6}
\renewcommand{\contentsname}{Sadržaj}

\lstset{
language=csh,
basicstyle=\footnotesize\ttfamily,
numbers=left,
numberstyle=\tiny,
numbersep=5pt,
frame=box,
tabsize=2,
extendedchars=true,
breaklines=true,
stringstyle=\color{blue}\ttfamily,
showspaces=false,
showtabs=false,
xleftmargin=17pt,
framexleftmargin=17pt,
framexrightmargin=5pt,
framexbottommargin=4pt,
commentstyle=\color{green},
morecomment=[l]{//}, %use comment-line-style!
morecomment=[s]{/*}{*/}, %for multiline comments
showstringspaces=false,
morekeywords={ abstract, event, new, struct,
as, explicit, null, switch,
base, extern, object, this,
bool, false, operator, throw,
break, finally, out, true,
byte, fixed, override, try,
case, float, params, typeof,
catch, for, private, uint,
char, foreach, protected, ulong,
checked, goto, public, unchecked,
class, if, readonly, unsafe,
const, implicit, ref, ushort,
continue, in, return, using,
decimal, int, sbyte, virtual,
default, interface, sealed, volatile,
delegate, internal, short, void,
do, is, sizeof, while,
double, lock, stackalloc,
else, long, static,
enum, namespace, string},
keywordstyle=\color{cyan},
identifierstyle=\color{black},
backgroundcolor=\color{white},
}

\title{Projektni zadatak - Konzistencija}
\author{Ognjen Čavić E2 161/2024}
\date{\today}
\begin{document}
\maketitle
\tableofcontents
\break
\section{Uvod}
Potrebno je napraviti sistem u kojem 10 senzora na nasumičnim vremenskim itervalima
generišu nasumične vrednosti, dok "menadžer" vrši njihovo poravnjanje, tj. bira
jednu od tih vrednosti koja će biti proglašena glavnom.
Ovaj sistem se može opisati kao skup od 3 glavne komponente:
\begin{itemize}
	\item \textbf{Simulator} - Pokreće po jednu nit za svaki senzor.
		Svaka nit na vremenskom intervalu između 1 i 10 sekundi generiše novu
		vrednost i  upisuje je u sopstvenu tabelu u bazi podataka pomoću 
		\textit{Entity Framework}-a.
	\item \textbf{Menadžer}- Na svakih 60 sekundi vrši poravnanje, tako što bira 
		čitanje senzora koje je najpovoljnije i upisuje u tabele svih senzora.
		Kriterijum za izbor najbolje vrednosti je najskorije očitavanje u 
		prihvatljivom opsegu $\pm5^oC$.
		Ukoliko takvo ne postoji uzima se najskorije.
	\item \textbf{WCF servis} - Definiše način komunikacije između menadžera i
		senzora.
\end{itemize}
\section{Organizacija rešenja}
Rešenje se sastoji od 4 manja projekta, gde svaki predstavlja jednu od prethodno
navedenih celina, sa tim da dodati četvrti projekat opisuje način
struktuiranja podataka za upis u bazu podataka:
\begin{itemize}
	\item \textbf{ConsistencyService} - WCF servis
	\item \textbf{ConsistencyManager} - menadžer koji vrši poravnanje
	\item \textbf{Sensors} - pokretanje niti koje simuliraju senzore
	\item \textbf{SensorsDatabaseContext} - organizacija baze podataka
\end{itemize}
\subsection{ConsistencyService}
\subsubsection{ConsistencyService/ConsistencyService.cs} \label{sec:servicecs}
Klasa \textbf{TemperatureInfo} opisuje oblik podataka koji se razmenjuju između
korisnika servisa i formiraju ga 4 polja:
\begin{itemize}
	\item \textbf{sensor\_id} - Identifikacioni broj senzora, broj od 1 do 10 
	\item \textbf{temperature} - Vrednost temperature
	\item \textbf{timestamp} - vremenski podatak
	\item \textbf{source} - izvor podatka, tj. ko ga upisuje u bazu, senzor ili
		menadžer tokom poravnanja
\end{itemize}
U ovom fajlu su samo još definisani interfejsi servisa. \textbf{IPublisher}
definiše metod za upis u bazu, dok interfejs \textbf{ISubscriber} definiše metod
za slanje upita.
\subsubsection{ConsistencyService/ConsistencyService.svc.cs}
Implementacija \textbf{IPublisher} i \textbf{ISubscriber} interfejsa unutar
\textbf{ConsistencyService} klase.
Bitno je primetiti da klasa nasleđuje oba interfejsa.
Metoda \textbf{publishTemp} upisuje zadati \textbf{TemperatureInfo}, što se
koristi da i menadžer i senzori upisuju u bazu, tako što \textbf{source} polje
postave na odgovarajuću vrednost.
Poslednje očitavanje senzora se dobija pozivom \textbf{querySensor} i
prosleđivanjem odgovarajućeg rednog broja.
Obe ove funkcije vrše odgovarajuću tranformaciju tipova tako da korisnici
servisa mogu da pozivaju operacije razmene informacija sa bazom podataka bez
poznavanja unutrašnje organizacije u istoj.
\subsection{ConsistencyManager}
\subsubsection{ConsistencyManager/Program.cs}
Sadrži dve funkcije, \textbf{Main} i \textbf{findBest}.
\textbf{Main} funkcija na svakih 60 sekundi šalje upit bazi podataka za 
poslednji podatak iz svake tabele, potom pomoću funkcije \textbf{findBest}
određuje koje od čitanja senzora je bilo najbolje.
Nakon toga se izmenjuju polja \textbf{source} i \textbf{timestamp} tako da se
označi da je podatak upisan kao rezultat poravnanja i vreme kada je on upisan,
takav podatak se upisuje u svaku od tabela.
Potrebno je napomenuti da menadžer instancira \textit{subscribera} i 
\textit{publishera}, koje koristi za upite i upise u bazu, respektivno.
\subsection{Sensors}
\subsubsection{Sensors/Program.cs}
Svrha ovog fajla je da instancira 10 objekata klase \textbf{Sensors} i pokreće
10 niti u vidu taskova.
\subsubsection{Sensors/Sensor.cs}
Definicija klase \textbf{Sensor}, koja sadrži 5 ali praktično 4 polja.
\begin{itemize}
	\item \textbf{sensorID} - redni broj senzora
	\item \textbf{\_random} - statičko polje za generator nasumičnih brojeva
	\item \textbf{maxTimeDelay} - maksimalni vremenski razmak do generisanja
		sledeće vrednosti
	\item \textbf{random} - generator koji je u konstrukturu postavljen na
		statički generator
	\item \textbf{publisherClient} - klijent servisa koji će pisati u bazu
		podataka
	\item \textbf{\_randomLock} - objekat koji se koristi za zaključavanje
		dela koda koji generiše nasumične brojeve.
\end{itemize}
Polje \textbf{random} je postavljeno da bude jednako statičkom \textbf{\_random}
zato što ako svaka instanca senzora ima svoj generator, desiće se to da imam 10
generatora nasumičnih brojeva sa vrlo sličnim "semenom" tj. \textbf{seed-om}.
To znači da će sekvence generisanja brojeva za sve senzore biti identične zato
što se implicitno koristi seed koji zavisi od vremena inicijalizacije.
Međutim, pošto objekat koji pravi nasumične brojeve može da pukne ukoliko ga
više od jedne niti koristi u bilo kom trenutku, potrebno je zaključati deo koda
koji poziva metode tog objekta.
\subsection{SensorsDatabaseContext}
\subsubsection{SensorsDatabaseContext/DbContext.cs}
Definicija kako su organizovani podaci unutar baze podataka.
Klasa \textbf{DBInfo\_Sensor} opisuje kako izgleda jedan red baze, odnosno kako
se jedno očitavanje senzora predstavlja unutar baze i sadrži 4 polja:
\begin{itemize}
	\item \textbf{id} - redni broj podatka unutar tabele, nije povezano sa
		rednim brojem senzora
	\item \textbf{temperature} - temperatura koju je senzor očitao
	\item \textbf{timestamp} - vreme očitavanja
	\item \textbf{source} - ko je upisao podatak, senzor ili menadžer
\end{itemize}
Pošto je potrebno imati odvojenu tabelu za svaki od senzora, a \textit{Entity 
Framework} zahteva da svaka tabela ima posebno definisan tip, jednostavno je
napravljeno dodatnih 10 klasa \textbf{DBInfo\_Sensor1}, ... , \textbf{DBInfo\_
Sensor10} i svaka od njih nasleđuje  \textbf{DBInfo\_Sensor}.
\textbf{SensorsDBContext} je izveden iz \textbf{DbContext} i sadrži 10 polja,
gde svako od njih predstavlja po jednu od tabela unutar baze podataka.
\section{Rezultati}
\subsection{Pokretanje}
Za pravilno funkcionisanje napisanog koda, potrebno je pokrenuti projekte
(desnim klikom na projekat \textit{Debug}$\rightarrow$\textit{Start new instance}) 
u sledećem redosledu:
\begin{enumerate}
	\item \textbf{ConsistencyService} - WCF servis
	\item \textbf{Sensors} - pokretanje niti koje simuliraju senzore
	\item \textbf{ConsistencyManager} - menadžer koji vrši poravnanje
\end{enumerate}
Jedna olakšica jeste to što visual studio ima opciju da se na start dugme
podigne proizvoljan broj projekata u proizvoljnom redosledu.
Desnim klikom na naziv celog rešenja tj. \textbf{solution-a}, i odabirom opcije
properties otvara se prozor gde je potrebno sa leve strane odabrati \textbf{Configure
Startup Projects} i konfigurisati ga kao na slici:
\begin{figure}[H]
	\centering
	\includegraphics[width=0.9\linewidth]{figs/Screenshot 2025-11-26 123940.png}
	\caption*{Slika 1: Podešavanje pokretanja projekta}
\end{figure}
\subsection{Rezultati}
Nakon pokretanja, otvaraju se dva prozora.
Na jednom prozoru se ispisuju poruke o tome koji senzor je očitao koju vrednost,
dok na drugom prozoru se  ispisuju poruke menadžera nakon što izvrši poravnanje.
\begin{figure}[H]
	\centering
	\includegraphics[width=0.9\linewidth]{figs/rezultat_1.png}
	\caption*{Slika 2: Prozori sa porukama o merenjima senzora i izvršenim
	poravnanjima menadžera}
\end{figure}
Druga stvar koju je potrebno proveriti je da li su podaci upisani u bazu podataka
i da li su validni.
Na slici 3 se može videti da se vremenski sve poklapa, tj. da senzor upisuje u
svoju tabelu na nasumničnom intervalu između 1 i 10 sekundi, dok menadžer
upisuje poravnatu vrednost svakog minuta.
\begin{figure}[ht!]
	\centering
	\includegraphics[width=0.9\linewidth]{figs/rezultat_2.png}
	\caption*{Slika 3: Prikaz redova u bazi podataka}
\end{figure}
\end{document}

